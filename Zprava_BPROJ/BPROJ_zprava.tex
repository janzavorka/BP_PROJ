\documentclass[a4paper,12pt, twoside]{article} % papír A4, písmo 12 bodu
\usepackage[utf8x]{inputenc} %kodovaní UTF-8
\usepackage{ucs} %kodovani unicode
\usepackage[czech]{babel} %podpora cestiny
\usepackage[T1]{fontenc} %pouzij variantu pisma T1 (hacky, carky)
%\usepackage[left=3.5cm,right=2cm,top=2.5cm,bottom=2.5cm]{geometry} %okraje stranky
\usepackage{amsmath,amsfonts,amssymb} %podpora matematiky
\usepackage{gensymb,marvosym} %symboly celsius (\celsius) apod.
\usepackage{times} %font Times New Roman (matematika bude vychozim pismem Computer Modern) 
\clubpenalty 10000  %kontrolovat sirotky
\widowpenalty 10000 %kontrolovat vdovy
\usepackage{setspace} \onehalfspacing %podpora pro zmenu radkovani + radkovani 1,5
\usepackage{enumerate} %podpora pro zmenu cislovani
\usepackage[unicode]{hyperref} %podpora hypertextu
\usepackage{fancyhdr} %vlastni zahlavi a zapati
\usepackage{graphicx} %podpora grafiky
\graphicspath{{img/}} %vychozi adresar s obrazky
\usepackage[normalem]{ulem} %Pro tabulku z Tablesgenerator.com
\useunder{\uline}{\ul}{} %Pro tabulku z Tablesgenerator.com
%--prostredi pro vkladani grafu ---
\usepackage{float}
\newfloat{graf}{hbtp}{ext}
\floatname{graf}{Graf}
%---pouziti:
%	\begin{graf}[hbtp]
%	\includegraphics[width=10cm]{graf.png}
%	\caption{Popisek grafu}
%	\end{graf}
%


%Redefinice prostředí abstract
\renewenvironment{abstract}
 {\small
  \begin{center}
  \bfseries \large \abstractname\vspace{-.5em}\vspace{0pt}
  \end{center}
  \list{}{%
  \itshape
    \setlength{\leftmargin}{0mm}
    \setlength{\rightmargin}{\leftmargin}%
  }%
  \item\relax}
 {\endlist}
 
%Prostředí pro klíčová slova
\newenvironment{keywords}
 {\small
 \vspace*{1cm}
  \list{Klíčová slova:}{
    \setlength{\leftmargin}{1cm}%
    \setlength{\rightmargin}{0cm}%
  }%
  \item\relax}
 {\endlist}



%--------------- UDAJE O PRACI -------
\newcommand{\jmeno}{Jan Závorka} %Jméno autora 
\newcommand{\nazev}{Arduino - piškvorky} %Název práce (čeština)
\newcommand{\nazevEN}{Arduino - tic-tac-toe}
\newcommand{\workType}{Projekt bakalářský} %Typ práce
\newcommand{\fakulta}{Fakulta elektrotechnická} %Fakulta
\newcommand{\katedra}{Katedra radioelektroniky} %Katedra
\newcommand{\vedouci}{Ing. Stanislav Vítek, Ph.D.} %Vedoucí práce
\newcommand{\datum}{\today} %Případně místo \today vložit svoje napevno 
%-------------------------------------


\begin{document}
%>>>>>>> Titulní strana
\begin{titlepage}
\centering
\includegraphics[scale=0.6]{/titulka/ctu_logo_blue.pdf}
\begin{center}
\vspace*{1cm}{\Large \bf ČESKÉ VYSOKÉ UČENÍ TECHNICKÉ V PRAZE}
{\large \bf \fakulta} \\
{\large \bf \katedra} \\
\vspace*{2cm} {\LARGE {\bf \nazev}}\\
\vspace*{0.7cm}
{\LARGE {\bf \nazevEN}} \\
\vspace*{1.cm}{\large \workType}
\end{center}

~\vfill
\begin{tabular}{p{.2\linewidth} p{.5\linewidth} p{.25\linewidth}}
~ & ~ & ~ \\
Studijní obor: & {\bf Elektronika a komunikace} & ~\\[.4em]
{ Vedoucí práce:} & \multicolumn{2}{l}{{\bf \vedouci} } \\
\end{tabular}\\
~\vfill
\begin{center}
{\Large \bf \jmeno} \\
{\Large \bf Praha, \datum}
\end{center}
\end{titlepage}
%>>>>>>> Prohlášení
\setcounter{page}{0} %cislo strany
\pagestyle{empty} %Nezobrazovat číslo stránky
\newpage ~\vfill „Prohlašuji, že jsem předloženou práci vypracoval samostatně a že jsem uvedl
veškeré použité informační zdroje v souladu s Metodickým pokynem o
dodržování etických principů při přípravě vysokoškolských závěrečných prací.“\\[3em]
V~Praze dne \today \hspace{.2\textwidth} \dotfill\\
\hspace*{11cm} Podpis
%>>>>> ABSTRACT <<<<<<
\newpage
\begin{abstract}
%----- TADY BUDE ABSTRACT -----
\end{abstract}
%>>>>>>> KEYWORDS <<<<<<<
\begin{keywords}
Arduino ethernet, piškvorky, komunikace Arduino-Arduino
\end{keywords}
%>>>>> Obsah <<<<<<<
\newpage
\tableofcontents %vytiskne obsah
%>>>>>>> Seznam tabulek a obrázků <<<<<<<
\newpage
\listoftables
\listoffigures
% --- definice zapati a cislovani ---
\newpage 
\pagestyle{fancy} %vlastni zahlavi/zapati
\renewcommand{\headrulewidth}{0pt} %bez linky v zahlavi
\renewcommand{\footrulewidth}{.5pt} %linka v zapati
\lhead{}       \chead{} \rhead{} %pole zahlavi (prazdna)
\lfoot{\nazev} \cfoot{} \rfoot{\thepage} %pole zapati
%
%>>>>>>> Vlatní text práce <<<<<<<
\clearpage
\section{Úvod} 
\label{kap:uvod}
Práce se
%
%
\clearpage
\section{Závěr}
%
%>>>>>>> Seznam literatury <<<<<<<
\clearpage
%--- Změna nadpisu literatury
\renewcommand{\refname}{Seznam použité literatury a zdrojů informací}
\addcontentsline{toc}{section}{Seznam použité literatury a zdrojů informací}
\begin{thebibliography}{99}
%
\end{thebibliography}
%--- SEZNAM SOFTWARU ---
\clearpage
\phantomsection %pridej odkaz do PDF zalozek
\addcontentsline{toc}{section}{Seznam použitého softwaru}
\section*{Seznam použitého softwaru}
\begin{enumerate}%[--]
	\item \TeX maker, \TeX Live
	\item \href{https://www.arduino.cc/en/main/software}{Arduino IDE}
	\item \href{https://www.tablesgenerator.com/latex_tables}{Tables Generator}
	\item \href{https://www.citace.com/citace-pro}{Citace.com}
	\item \href{https://easyeda.com/}{EasyEDA}
	\item Linux Mint 18.1 Cinnamon 64-bit
\end{enumerate}
%--- SEZNAM PRILOH ---
\phantomsection %pridej odkaz do PDF zalozek
\addcontentsline{toc}{section}{Seznam příloh}
\section*{Seznam příloh} 
\begin{enumerate}[{Příloha} 1:]
\item příloha
\end{enumerate}
\end{document}