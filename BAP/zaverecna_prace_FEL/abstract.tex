\section*{Abstrakt}
Tato bakalářská práce se zabývá vývojem a výrobou jednoduchého zařízení, které umožňuje demostrovat využití energeticky nenáročných zařízení v IoT sítích. Zařízení má podobu jednoduché hry pro více hráčů - piškvorek a je postaveno na platdormě Arduino. Zařízení je realizováno pomocí tří koncových zařízení (klientů) s dotykovými displeji pro interakci s uživatelem a jednoho centrálního řídicího prvku (serveru). V práci je popsán konkrétní použitý hardware včetně návrhu krabiček. Dále je zde podrobně rozepsán vytvořený software včetně možnosti úprav pro použití s jinými moduly. Nakonec je uveden i návod na oživení a obsluhu.\\

\vspace{.5cm}
\noindent
Klíčová slova: Arduino Ethernet, IoT demonstrátor, Arduino hra, Arduino IoT



\section*{Abstract}
{
\selectlanguage{english}
This bachelor thesis deals with development and production simple device which can demostrate usage  of low power devices in IoT networks. This device has form of multiplayer game - Noughts and crosses and is based on Arduino platform. The device has three end nodes (clients) with touch screen for interaction with user and one control device (server). There is descriped used hardware including design of cases for all devices in this thesis. There is also description of software including list of possible changes which could be made for purpose to use this product with different modules. In the end there is manual for starting and operating this device.

\vspace{.5cm}
\noindent
Key words: Arduino Ethernet, IoT demonstration device, Game based on Arduino, Arduino IoT
}
