\section{Úvod}
Cílem práce bylo vytvořit zařízení, které by demonstrovalo využití jednodeskových počítačů v IoT sítích. Pro realizaci byla zvolena platforma Arduino. Důvodem pro zvolení této platformy bylo především její rozšíření mezi uživateli a dostupnost doplňkových periferií. Samotné zařízení by mělo mít funkci hry pro více hráčů. Koncová zařízení budou vybavena budou vybavena dotykovým displejem pro interakci s uživatelem. Celá hra pak bude řízena jednou centrální jednotkou taktéž založenou na platformě Arduino. Komunikace mezi jednotlivými zařízeními pak bude probíhat po Ethernetové síti. Pro dobrou názornost a nenáročnost (co se týče složitosti implementace, tak i potřebného hardwarového výkonu) byla zvolena hra piškvorky.

Práce je rozdělena do tří částí. V první části je popsán použitý hardware, včetně jeho případných úprav. Nechybí zde ani popis dalších komponent, které jsou potřeba pro kompletaci celého zařízení. Druhá část je zaměřena na popis softwaru, především na metody komunikace koncových zařízení. Pro názornost jsou některé důležité úkony doplněny diagramy. Poslední část je věnována už samotnému funkčnímu zařízení. Lze zde nalézt návod na zprovoznění, včetně fotografií funkčního prototypu.

%{\color{ashgrey} Nějaké obecné věci o IoT}

Samotný pojem IoT (\uv{Internet of things} - Internet věcí) označuje propojení fyzických zařízení prostřednictvím internetu. Jako koncová fyzická zařízení jsou většinou používány různé vestavěné systémy vybavedené samotným řídicím mikrokontrolérem a řadou senzorů. Tato zařízení dokáží komunikovat jak mezi sebou, tak se zeřízením uživatele (mobilním telefon) nebo dokáží předávat naměřená data na cloud.

IoT samozřejmě není jen o koncových zařízeních, dalším důležitým prvkem jsou brány (\textit{gateway}), které zajišťují spojení mezi koncovými zařízeními a například cloudem. IoT síť lze rozdělit do několika kategorií \cite{iot_geographic} podle fyzického rozmístění:
\begin{itemize}
  \item \textit{nanonetwork} \ldots spojení několika malých (řádově mikrometrů) zařízení, které mají za úkol plnit jednoduché úkony.
  \item \textit{NFC (Near-Field Communication)} \ldots spojení zařízení na vzdálennost řádově jednotky centimetrů.
  \item \textit{BAN (Body Area Network)} \ldots spojení zařízení v oblasti lidského těla, zejména různá nositelná zařízení případně senzory uvnitř těla.
  \item \textit{PAN (Personal Area Network)} \ldots síť v oblasti jedné místnosti.
  \item \textit{LAN (Local Area Network)} \ldots síť v oblasti jedné budovy.
  \item \textit{CAN (Corporate Area Network)} \ldots síť v oblasti jedno kampusu/společnosti, spojuje několik lokální sítí.
  \item \textit{MAN (Metropolitan Area Network)} \ldots síť v oblasti jedno města.
  \item \textit{WAN (Wide Area Network)} \ldots síť pokryvající větší geografickou oblast, spojuje menší sítě.
\end{itemize}
Z tohoto pohledu spadá vytvořená platforma do sítí typu \textit{LAN}. Teoreticky by bylo možné ho připojit například do internetu, problém však je, že zařízení na to není stavěné - nemá implementovány žádné ochranné/šifrovací mechanismy ani žádné autentizační procesy. Díky hardwarové implementaci TCP/IP v kontrolérech WIZnet~\cite{datasheet_w5100} by nemělo docházet k výraznému ovlivnění útoky typu DDoS \cite{ArdIotDDOS}.

Další důležitou součástí IoT je i implementace nových protokolů a standardů pro komunikaci. Mezi nejznámnější lze například zařadit úspornou verzi Bluetooth - \textit{BLE: Bluetooth low energy}, které se používá například u nositelných zařízeních. Dalším, spíše na průmysl zaměřeným, protokolem je \textit{ZigBee}, který je známý díky svému dobrému zabezpečení. Pro použití v rozlehlých sítích s velkým počtem zařízení je vhodný LoRaWAN \cite{IoTprotocols}. I když je na výběr v velké palety  protokolů (lišící se například přenosovými rychlostmi, dosahem), byl pro komunikaci mezi zařízeními vybrán Ethernet.
