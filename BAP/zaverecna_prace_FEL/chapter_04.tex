\section{Závěr}
Podařislo se sestrojit zařízení, které zábavnou formou demonstruje možnosti komunikace nízkoenergetický jednočipových počítačů (Arduin) po síti a tím i prezentovat možnosti komunikace v IoT sítí, kdy mezi jednotlivými zařízeními není potřeba přenášet velké množství dat.

Během realizace projektu se vyskytlo několik problémů, některé z nich bylo možné úplně eliminovat, jiné vedly k určitým kompromisům. Všechny tyto problémy jsou pak zmíněny v této práci včetně zvoleného řešení. I když zařízení bylo během vývoje pravidelně testováno, nevylučuje se, že by mohlo obsahovat chyby. Opravené kódy pak budou zveřejňovány na autorově \href{https://github.com/janzavorka/BP_PROJ}{GitHubu} \footnote{Adresa: https://github.com/janzavorka/BP\_PROJ}, kde má celý projekt od počátku vytvořenou stránku. Kromě všech zdrojových kódů, lze na ní nalézt i tuto práci a další materiály, například soubory krabiček (stl soubory i soubory pro program Autodesk Inventor). Krom oprav chyb zde budou umístěna i případná vylepšění.
