\section{Popis softwaru}
\subsection{Knihovny}
\label{sec:knihovny}
Pro realizaci byly použity následující knihovny:
\begin{itemize}
  \item \textit{Ethernet library}: knihovna slouží pro připojení Arduina k internetu. Jak je uvedeno na webových stránkách \cite{EthLib} jsou podporovány desky a shieldy založené na konrolérech W5100, W5200, W5500. Tato knihvna je použita jak u serveru tak i u klientů. U serveru připojený ethernet shield obsadí piny 10, 11, 12, 13. Komunikace Arduina se shieldem (ethernet kontrolérem) probíhá po SPI sběrnici.

  \item \textit{MCUFRIEND\_kbv library}: knihovna slouží k ovládání displeje u klienta. Pro správnou funkci je nutné, jak se zmiňuje autor na domovské stránce knihovny~\cite{lib_MCUFRIEND_kbv},  mít k dispozici také knihovnu \textit{Adafruit\_GFX} \cite{lib_adafruitGFX}. Pro použití displeje se nejdříve vytvoří instance třídy \texttt{UTFGLUE}, kde je nutné správně definovat piny pro daný shield, v tomto případě \texttt{UTFTGLUE LCD(0x0154,A2,A1,A3,A4,A0);}. Pro řízení displeje jsou pak volány patřičné metody formou \textit{LCD.metoda(parametry);}, například pro vyplnění celého displeje černou \texttt{LCD.clrScr();}. Pomocí ukázkouvého zdrojového kódu z této knihovny \textit{(examples/TouchScreen\_
calibr\_kbv)} byly získány kalibrační hodnoty pro dotykovou plochu displeje.

  \item \textit{Adafruit\_TouchScreen}: slouží k zaznamenání hodnot ze čtyřvodičové dotykové plochy. Stejně jako u knihovny pro displej je nejdříve nutné vytvořit instanci dané třídy \texttt{TouchScreen} v tomto případě tak \texttt{TouchScreen Touch(XP, YP, XM, YM, 300);}, kde \textit{XP, YP, XM, YM} jsou piny, na které je vyvedená dotyková plocha (pro použitý shield jsou to: XP = 6, YP = A1, XM = A2, YM = 7). Hodnota 300 by pak dle popisu knihovny \cite{lib_touch} označuje elektrický odpor dotykové plochy X (měřeno multimetrem mezi piny XP a XM). Pro použité shiledy je hodnota přibližně $300  \ \mathrm{\Omega}$. Dále je třeba určit minimální a maximální tlak pro vyhodnocení dotyku pomocí \texttt{\#define MINPRESSURE} a \texttt{\#define MAXPRESSURE}. Vyhovující jsou hodnoty uvedené v ukázkových kódech pro knihovnu a to konkrétně 10 pro \texttt{MINPRESSURE} a 1000 pro \texttt{MAXPRESSURE}.

  \item \textit{SimpleTimer Library for Arduino}: jednoduchá knihovna, která slouží k řízení určitých časových událostí (kde není třeba velká přesnot), například pro zobrazení a skrytí chybových hlášek, blikání indikační svítivé diody u serveru. Dle webových stránek \cite{lib_simpleTimer} je knihovna založená na funkci \texttt{milis();} (vrací počet milisekund od začátku běhu programu) a nevyužívá přerušení ani harwarový timer.
\end{itemize}

\subsection{Konfigurační hodnoty}
V této kapitole jsou popsány jednotlivé konfigurační hodnoty, která by mohlo být třeba upravit pro správnou funkci celého zařízení. Změny se vždy provádí přímo ve zdrojovému kódu, vždy pouze v hlavní souboru: \textit{piskvorky\_MP\_client.ino} pro klienta a \textit{piskvorky\_MP\_server.ino} pro server. Jak už bylo zmíněno v kapitole \ref{sec:HWclient}, nebylo možné, kvůli malé paměti pro program, implementovat menu, proto je nutné změny provádět přímo ve zdrojovém kódu a dané zařízení pak přeprogramovat. V případě serveru je k dispozici microUSB konektor, který je dostupný skrze obdélníkový otvor na boku zařízení. V případě clienta je nutné demontovat víko, odpojit napájecí modul (v modré bužírce) a připojit Arduino k počítači pomocí převodníku USB - UART.

Jednotlivé konfigurační bloky jsou vždy ohraničeny komentáři mezi nimiž se nacházejí jednotlivé proměnné a popis jejich funkce a omezení hodnot. Níže je uveden příklad ohraničení bloku pro nastavení sítě (IP adresa a port):
\begin{verbatim}
/* ---------- KONFIGURACE - nastavení sítě ----------*/

/* ---------- KONEC - nastavení sítě ----------*/
\end{verbatim}


\subsubsection{Konfigurace serveru}
\label{sec:server-nastaveni}
%Nastavení sítě
V prvním případě je nutné vybrav síťový režim, v případě \textit{ETHMODE\_STATIC} je použita IP adresa, která je uložena v proměnné \textit{serverAddress}. V případě volby \textit{ETHMODE\_DHCP} je adresa přiřazena DHCP serverem a je nutné dodržet aby se přiřazená adresa neměnila a byla zároveň nastavená u jednotlivých klientů.

Dále je třeba nastavit příslušnou MAC adresu, v případě použitého nebyla přiřazena výrobcem, takže je nutné nějakou zvolit. Vzhledem k tomu, že zařízení se provozuje na samostatné lokální síti, byla MAC adresa vybrána náhodně, v případě, že v síti jsou další zařízení, je možné vybrat MAC adresu například podle postupu popsaného v \cite{vyberMAC}. Vezme se MAC adresa nějaké zařízení v síti a poslední bajt se zvětší o jedna a takto vzniklá MAC adresa se přiřadí shieldu.
Vybraná MAC adresa je v poli \texttt{mac}.

Jako poslední je potřeba přiřadit port, na kterém budou zařízení komunikovat. Dle normy \cite{norm_RFC6335} je možné zvolit jakoukýkoliv dynamický port (49152-65535), protože tyto porty nebudou nikdy přiřazeny žádné službě. Port je uložen v proměnné \texttt{localPort} a stejný port musí být nastaven i u clientů.
\lstinputlisting[language=C++, style=customc_config, firstline=1, lastline=8, title= Blok konfigurace ethernet shieldu pro server]{codes/server_config.cpp}

%Nastavení pinů
Jak je popsáno v kapitole \ref{sec:HWserver}, jsou k serveru připojena dvě tlačítka a jedna barevná svítivá dioda. V případě, že je nutné tyto periferie připojit jinak, je možné změnit čísla pinů v kofiguračním bloku \textit{nastavení pinů a LED}. Tlačítka využívají interní pull-up rezistory a jejich zapojení je uvedeno na obrázku \ref{fig:server_module}.
\lstinputlisting[language=C++, style=customc_config, firstline=10, lastline=22, title= Blok konfigurace pinů pro svítivou diodu a tlačítka]{codes/server_config.cpp}

%Nastavení barev
Jako poslední je možné nastavit jakou barvu budou mít jednotliví klienti. V ukázkové případě jsou barvy přiřazeny z výběru, který je uveden v kódu, ale je možné také definovat vlastní ve formátu RGB565 (16-bit barva).
\lstinputlisting[language=C++, style=customc_config, firstline=24, lastline=30, title= Blok konfigurace jednotlivých barev pro klienty/hráče]{codes/server_config.cpp}

\subsubsection{Konfigurace klienta}
\label{sec:client-nastaveni}
%Nastavení profilů
Klientů se v celé sestavě nachází několik (v tomto případě tři) a liší se pouze určitým nastavením (IP adresa, MAC adresa, kalibrační hodnoty displeje) proto jsou vytvořeny profily pro jednotlivé klienty (softwarově omezeno na pět). Každému profilu se nastaví požadované parametry a při nahrávání programu do Arduina se pak profily mění pomocí \texttt{\#define CLIENTx}, kde x je číslo jednotlivých klientů/profilů. Pokud není zvolen žádný profil, použití je defaultní hodnoty a uživatel je o tom informován při překladu pomocí direktivy \texttt{\#warning} (zobrazí se hlášení, ale překlad neukončí).
\lstinputlisting[language=C++, style=customc_config, firstline=1, lastline=3, title= Blok výběru profilu pro klienta]{codes/client_config.cpp}

%Nastavení MAC adres
V této části se přiřazují jednotlivé MAC adresy daným profilům. Výhodou oficiálních Arduino Ethernet desek je, že mají MAC adresu přidělenou výrobcem a je možné jí nalézt na bílém štítku ze spodní strany.
\lstinputlisting[language=C++, style=customc_config, firstline=6, lastline=38, title= Blok přiřazení MAC adres jednotlivým klientským profilům]{codes/client_config.cpp}

%Nastavení IP adres
Stejně jako v případě serveru i zde je možné vybrat ze dvou režimů sítě a to \textit{ETHMODE\_STATIC} a \textit{ETHMODE\_DHCP}. V případě volby \textit{ETHMODE\_DHCP} není na adresu přiřazenou DHCP serverem kladen žádný zvláštní nárok. V případě použití \textit{ETHMODE\_STATIC} je ještě nutné dodefinovat IP adresy pro jednotlivé klienty. Opět lze s výhodou využít profilů, pokud se použije defaultní, je o tom uživatel opět při překladu informován.
Kromě toho je ještě nutné doplnit IP adresu serveru, ke kterému se budou klienti připojovat. Ta byla nastavena při konfiguraci serveru (kapitola \ref{sec:server-nastaveni}) stejně tak jako port, který je také nutné zvolit stejný.
\lstinputlisting[language=C++, style=customc_config, firstline=40, lastline=66, title= Blok nastavení síťového režimu a IP adres]{codes/client_config.cpp}

%Nastavení kalibrace displeju
Jako poslední je nutné nastavi kalibrační hodnoty dotykové plochy displeje. Nejjednodušší způsob jejich získání je použít ukázkový program v knihovně \textit{MCUFRIEND\_kbv library}, který lze nalézt  v \textit{(examples/TouchScreen\_
calibr\_kbv)}. V ukázkovém programu je třeba upravit podle použitého shiledu nastavení pinů. Poté stačí postupovat podle pokynů na displeji a výsledek se vypíše na sériovou linku (lze použítintegrovaný v Arduino IDE - \textit{Tools->Serial Monitor}). Krom kalibračních hodnot je také nutné doplnit orientaci displeje (\texttt{\#define TOUCH\_LANDSCAPE} nebo \texttt{\#define TOUCH\_PORTRAIT}), v případě, že nebude zvolena ani jedna možnou, překlad budou ukončen a vypsána chybová hláška (direktiva \texttt{\#error}). I v tomto případě lze s výhodou využít profilů.
\lstinputlisting[language=C++, style=customc_config, firstline=68, lastline=109, title= Blok nastavení kalibračních hodnot dotykové plochy]{codes/client_config.cpp}

\subsection{Komunikace server - klient}
\label{sec:comm_server-client}
Server komunikuje s klienty pomocí pole \textit{board}. Jedná se o jednorozměrné pole typu \textit{byte} o velikosti 136 (velikost musí být dělitelná osmi). V tomto poli jsou vyplněné všechny důležité informace o stavu hry i jednotlivých hráčích, význam jednotlivých hodnot v poli \textit{board} je popsán v tabulce \ref{tab:packet_server-client}.

Při odesílání je toto pole rozděleno na části po osmi bajtech. Každá tato část je vybavena pořadovým číslem a dvoubajtovým kontrolním součtem (v něm je zahrnuto i pořadové číslo) a následně odeslána připojeným klientům.

Klienti postupně přijímají všechny části a v případě bezchybného příjmu data vyhodnotí. V případě, že klient na základě kontrolního součtu identifikoval chybnou část, odešle podle pravidel komunikace klient - server (popsáno v kapitole \ref{sec:comm_client-server}). Server pak v případě přijetí požadavku odešle danému klientovi vyžadovanou část pole.

\begin{table}[hbtp]
\catcode`\-=12
\caption{\label{tab:packet_server-client}Rozložení pole \textit{board} pro přenos dat a řízení hry mezi serverem a klientem}
\begin{tabular}{|c|c|l|}
\hline
Index   & Hodnoty     & \multicolumn{1}{c|}{Popis}                                      \\ \hline
0-89    &             & Každý index odpovídá jednomu čtverečku na herní desce piškvorek \\ \cline{2-3}
        & 0           & Pole je prázdné (neobsazené)                                    \\ \cline{2-3}
        & 1-5         & Obsazeno některým hráčem/klientem                               \\ \hline
90      &             & Přenos řídicí informace pomocí kódu                             \\ \cline{2-3}
        & 0           & nic nedělat                                                     \\ \cline{2-3}
        & 1           & Vše je OK, překreslit obrazovku                                 \\ \cline{2-3}
        & 3           & Příprava nové hry, zobrazit úvodní obrazovku                    \\ \cline{2-3}
        & 9           & Informace pro hráč odpojen, že bude odpojen                     \\ \cline{2-3}
        & 100         & Hra skončila remízou                                            \\ \cline{2-3}
        & 10x         & Hodnota podle hráče, který vyhrál: 101-105 ('x' je číslo hráče) \\ \cline{2-3}
        & 20x         & Problémy/odpojení s daného hráče: 201-205 ('x' je číslo hráče)  \\ \hline
91      & 1-5         & Číslo hráče, který je na tahu, pokud je 0, nikdo nehraje        \\ \hline
93      & 0-89        & Počet odehraných kol (vyplňuje server)                          \\ \hline
95-96   &             & Barva hráče 1                                                   \\ \cline{1-1} \cline{3-3}
97-98   &             & Barva hráče 2                                                   \\ \cline{1-1} \cline{3-3}
99-100  & Kód barvy   & Barva hráče 3                                                   \\ \cline{1-1} \cline{3-3}
101-102 &             & Barva hráče 4                                                   \\ \cline{1-1} \cline{3-3}
103-104 &             & Barva hráče 5                                                   \\ \hline
105-108 &             & IP adresa hráče 1                                               \\ \cline{1-1} \cline{3-3}
109-112 &             & IP adresa hráče 2                                               \\ \cline{1-1} \cline{3-3}
113-116 & IPv4 adresa & IP adresa hráče 3                                               \\ \cline{1-1} \cline{3-3}
117-120 &             & IP adresa hráče 4                                               \\ \cline{1-1} \cline{3-3}
121-124 &             & IP adresa hráče 5                                               \\ \hline
\end{tabular}
\end{table}


\subsection{Komunikace klient - server}
\label{sec:comm_client-server}
Klient komunikuje se serverem, až na výjimku při sestavení spojení, prostřednictvím dvoubajtového pole. Popis jednotlivých částí pole je v tabulce \ref{tab:clietn-server}. Pro kontrolu přenosu se každá zpráva odešle třikrát. Na straně serveru se pak porovnají dvě ze tří přijatých zpráv a ostatní přijatá data se zahodí. Klient může poslat data maximálně jednou za sekundu (posláním dat se rozumí odeslání tří stejných zpráv).

Pokud při porovnání server zjistí, že přijeté zprávy nejsou stejné, odešle danému klientovi znovu herní desku. V případě, že byl daný hráč na tahu a přenos požadavku na vyplnění pole se nezdařil, je mu prostřednictvím znovuodeslání pole znovu aktivován tah.

V případě, že klient přijal od serveru chybná data, odešle požadavek ve tvaru \mbox{\{20, x\}}, kde \textit{x} je číslo chybně přijaté části pole. Pokud je toto serverem správně přiato, odešle požadovanou část znovu (dle pravide komunikace server  -klient, kapitola \ref{sec:comm_server-client})

\begin{table}[hbtp]
\catcode`\-=12
\caption{\label{tab:clietn-server} Rozložení packetu pro komunikaci klient - server}
\begin{tabular}{|c|c|l|}
\hline
Index & Hodnoty                         & \multicolumn{1}{c|}{Popis}                                                                                                                                                                                \\ \hline
0     & 10                              & \begin{tabular}[c]{@{}l@{}}Jedná se o informaci, že další přenesený bajt bude číslo vyplněné\\ herní pozice v poli \textit{board}    \end{tabular} \\ \cline{2-3}
      & 20                              & \begin{tabular}[c]{@{}l@{}}Žádost klienta o poslání dané části herního pole (board), v další \\ bajtu je číslo dané části \textit{boardu}   \end{tabular}   \\ \hline
1     & \multicolumn{1}{l|}{0-89, 0-16} & \begin{tabular}[c]{@{}l@{}}Podle hodnoty předchozího bajtu: index vyplněného \\ pole nebo pořadí packetu, který má být poslán znovu\end{tabular}                                                          \\ \hline
\end{tabular}
\end{table}



\subsection{Sestavení spojení}
\subsubsection{Příprava serveru}
Při spuštění serveru dojde nejprve k identifikaci ethernet kontroléru. V případě, že je identifikován kontrolér WIZnet W5200 nebo WIZnet W5500, je k dispozici funkce \texttt{Ethernet.linkstatus()}, která umožňuje získat informaci v fyzickém připojení kabelu. Pokud není kabel připojen, je to signalizováno blikáním signalizační svítivé diody červenou barvou (viz. tabulka \ref{tab:LED_man}) a program nepokračuje dále, dokud není kabel připojen.

V případě, že server získává adresu z DHCP serveru, je možné ještě ověřit správné připojení pomocí kontroly získané adresy. V knihovně Arduino Ethernet \cite{EthLib} je toto řešeno návratovou hodnotou funkce \texttt{Ethernet.begin(mac)} (v režimu DHCP má pouze jeden parametr a to lokální MAC adresu, jinak funkce \texttt{Ethernet.begin()} nemá návratovou hodnotu). Proces přípravy serveru je znázorněn na obrázku \ref{fig:flow:serverStartUp}.

Pokud je však nastaven mód se statickou IP adresou a je použit shield s ethernet kontrolérem například WIZnet W5100 (jako je tomu v této práci), není zde žádná vhodná možnost kontroly správného připojení. Jedna z možností by byla, že by se server pokusil poslat zprávu \textit{Echo Request} nějakému známému zařízení v síti a očekávat odpověď \textit{Echo Reply}. Ale vzhledem k tomu, že jako centrální prvek je v tomto zapojení zvolen switch a žádné další zařízení v síti není, jei tato možnost nepoužitelná.

V této fázi je příprava serveru dokončena, postup navázání spojení s clientem je popsán v následující kapitole \ref{sec:connection_communication}.

\begin{figure}
  \centering
  \begin{tikzpicture}[scale=2, node distance = 2cm, auto]
    % Place nodes
  \node (start) [decision] {Jaký je Ethernet kontrolér?};
  \node (link) [decision,below left of = start,text width = 6em, xshift = -5cm, yshift=-3cm] {Je kabel připojen?};


  \node (wait1) [block, below of = start, text width = 10em, yshift=-1cm, xshift = -1cm] {Čekej 1 sekundu.};
  \node (blink1) [block, below of = wait1, text width = 10em] {Upozorni uživatele blikáním svítivé diody};

  \node (mode) [decision, below of = start, yshift = -6cm] {Ethernetový mód?};

  \node (static) [block, below left of = mode, text width = 18em, yshift=-3cm, xshift = -4cm] {Spusť server \\ \texttt{Ethernet.begin(mac, serverAddress)}};

  \node (dhcp) [decision, below right of = mode, yshift=-2.5cm, text width = 10em, xshift = +1cm] {Spusť server \\ \texttt{Ethernet.begin(mac)} \\ Přiřazena IP adresa?};

  \node (finish) [startstop, below  of = mode, yshift=-7cm, text width = 10em] {Sever je připraven};


  \draw [arrow] (start) -| node[anchor=south west] {W5200, W5500} (link);

  \draw [line] (start.east) -| node[anchor=south east] {W5100} ([xshift = 2.5cm, yshift=0.7cm] mode.north);
  \draw [arrow] ([xshift = 2.5cm, yshift=0.7cm] mode.north) -| node[anchor=south east] {} ( mode.north);

  \draw [line] (link.south) -- node[anchor=east] {} ([xshift = 0cm, yshift=-.1cm] link.south);
  \draw [line] ([xshift = 0cm, yshift=-.1cm] link.south) -- node[anchor=south east] {ANO} ([xshift = 0cm, yshift=.7cm] mode.north);

  \draw [arrow] (link) -- node[anchor=south] {NE} (blink1);
  \draw [arrow] (blink1) -- node[anchor=south] {} (wait1);
  \draw [arrow] (wait1) -- node[anchor=south] {} ([yshift=.5cm] link.north);

  \draw [arrow] (mode.west) -| node[anchor=south] {ETHMODE\_STATIC} (static.north);

  \draw [arrow] (mode.east) -| node[anchor=south west] {ETHMODE\_DHCP} (dhcp.north);

  \draw [line] (dhcp.west) -| node[anchor=south east] {NE} ([yshift = 1.2cm] dhcp.west);
  \draw [arrow] ([yshift = 1.2cm] dhcp.west) -- node[anchor=east] {} ([yshift = 0.4cm] dhcp.north);

  \draw [line] (dhcp.south) -- node[anchor=west] {ANO} ([xshift = 0.5cm]finish.east);
  \draw [arrow] ([xshift = 0.5cm]finish.east) -- node[anchor=west] {}  (finish.east);

  \draw [line] (static.south) -- node[anchor=south] {} ([xshift = -1.6cm]finish.west);
  \draw [arrow] ([xshift = -1.6cm]finish.west) -- node[anchor=west] {}  (finish.west);


  \draw [Startarrow] ([yshift=.5cm]start.north) -- node[anchor=east] {} (start.north);
\end{tikzpicture}

  \caption{\label{fig:flow:serverStartUp} Diagram naznačující proces přípravy a zapntí serveru}
\end{figure}




\subsubsection{Příprava klient}
Příprava probíhá obdobně jako u serveru, ale vzhledem k tomu, že použité Arduino Ethernet desky mají kontrolér WIZnet W5100, je vynechána část s kontrolou připojení kabelu. Jediná možná detekce správného připojení připojení kabelu je tak pouze v režimu, kdy je IP adresa získávána z DHCP serveru. V takovém případě se na displeji vypíše zpráva \textit{Zkontrolujte pripojeni kabelu} a klient čeká dokud IP adresu neobdrží. V opačném případě je přepnuto na úvodní obrazovku.

V této fázi je klient připraven a čeká na příkaz uživatele připojit se k serveru.

\subsubsection{Sestavení pojení}
\label{sec:connection_communication}
Pokud nastavení a spuštění serveru proběhlo správně, server v nekonečné smyčce (\texttt{void loop()\{\}}) pravidelně ověřuje, zda nějaké zařízení nevyžaduje připojení. To je realizováno metodou z knihovny \cite{EthLib} - \texttt{server.accept()}. V případě příchozího spojení vratí tato metoda objekt typu client, v opačném případě \textit{NULL}. To je uloženo do proměnné \texttt{newClient}. Metoda \texttt{server.accept()} vrací data o daném příchozího spojení pouze jednou a proto je nutné je uložit do extra proměnné.

Připojení klienta spustí uživatel stisknutím příslušého tlačítka na displeji. Klient se v pravidelných intervalech pokouší o navázání spojení funkcí  \\ \mbox{\texttt{client.connect(serverAddress, localPort)}}. V případě, že je připojení k serveru úspěšné, pošle klient serveru zprávu \uv{100}. Tento kód informuje server, že se jedná k lienta pro piškvorky, pokud by server přijal jiná data, spojení zruší.

Pokud server zaznamená nové příchozí spojení, spustí kontrolu nového hráče. Celý proces připojení nového hráče je naznačen na obrázku \ref{fig:flow:newClientServer}.

\begin{figure}
  \centering
  \begin{tikzpicture}[scale=2, node distance = 2cm, auto]
    % Place nodes
  \node (start) [decision] {Běží hra?};
  \node (saving) [block, below of=start, text width=6cm, yshift=-0.5cm] {Ulož příchozího klienta do pomocné proměnné \texttt{newClient}.};
  \node (test1) [decision,below of=saving, yshift=-.5cm, text width = 12em] {Bylo do proměnné \texttt{newClient} něco uloženo?};
  \node (test2) [decision,below of=test1, yshift=-1.2cm, text width = 10em] {Je pro nového klienta volná pozice v poli \texttt{clients[]}?};
  \node (test3) [decision,below of=test2, yshift=-1.2cm, text width = 10em] {Je už připojen nějaký klient se stejnou IP adresou?};
  \node (test4) [decision,below of=test3, yshift=-1.2cm, text width = 7em] {Poslal nový klient správný kód (\uv{100})?};

  \node (not) [block, right of=start, text width=10em, xshift=5cm] {Nic nedělej.};
  \node (disconnect) [startstop, right of=test2, text width=15em, xshift=6cm] {Ukonči spojení s novým klientem (\texttt{newClient.stop()}).};
  \node (resend) [block, right of=test3, text width=15em, xshift=6cm] {Odešli aktivnímu klientovi (pole \texttt{clients[]}) herní pole \texttt{board[]} znovu.};
  \node (end) [startstop,right of=test4, text width=20em, xshift=6.5cm] {
      \begin{enumerate}
        \item Zařaď nového klienta do pole aktivních klientů (pole \texttt{clients[]}).
        \item Aktualizuj IP adresy v poli  \texttt{board[]}.
        \item Odešli všem aktivním klientům aktualizované pole \texttt{board[]}.
        \item Informuj uživatele o nově připojeném klientovi blikáním svítivou diodou.
      \end{enumerate}
  };

  \draw [arrow] (start) -- node[anchor=east] {NE} (saving);
  \draw [arrow] (start) -- node[anchor=south] {ANO} (not);
  \draw [arrow] (saving) -- node[anchor=east] {ANO} (test1);
  \draw [arrow] (test1) -- node[anchor=east] {ANO} (test2);
  \draw [arrow] (test2) -- node[anchor=east] {ANO} (test3);
  \draw [arrow] (test3) -- node[anchor=east] {NE} (test4);
  \draw [arrow] (test4) -- node[anchor=south] {ANO} (end);
  \draw [arrow] (test3) -- node[anchor=south] {ANO} (resend);
  \draw [arrow] (resend) -- node[anchor=east] {} (disconnect);
  \draw [arrow] (test2) -- node[anchor=south] {NE} (disconnect);
  \draw [arrow] (test1) -| node[anchor=south east] {NE} (not);

  \draw [line] (test4) -- node[anchor=south east] {NE} ([yshift=-1cm] test4.south);
  \draw [line] ([yshift=-1cm] test4.south) -| node[anchor=south east] {} ([xshift=1.0cm, yshift=0cm] disconnect.east);
  \draw [arrow] ([xshift=1.0cm, yshift=0cm] disconnect.east) -- node[anchor=south east] {NE} (disconnect);

  \draw [line] (not.north) -- node[anchor=south east] {} ([yshift=.5cm]not.north);
  \draw [arrow] ([yshift=.5cm]not.north) -- node[anchor=south east] {} ([yshift=.2cm]start.north);


  \draw [Startarrow] ([yshift=.5cm]start.north) -- node[anchor=east] {} (start.north);
\end{tikzpicture}

  \caption{\label{fig:flow:newClientServer} Diagram naznačující proces příjmu nového klienta na straně serveru}
\end{figure}

V této fázi se čeká na připojení dalších klientů. Pravidelně je také kontrolováno, zda se nějaký klient neodpojil. V případě, že ano, jsou jeho data smazána jak z pole \texttt{clients[]} tak z herního pole \textit{board} a o této skutečnosti jsou ostatní klienti informováni zprávou (odeslání pole \textit{board}) s kódem 20x, kde \textit{x} je číslo odpojeného klienta. Tato zpráva je zobrazena na displejích klientů po dobu 4000 ms (nastaveno v proměnné \texttt{clientErrMessageLast}).

Nyní se čeká na zahájení hry ze strany uživetele (stisk zeleného tlačítka na serveru nebo příkazemn přes sériovou linku, viz. kapitola \ref{sec:ovladani}).

\subsection{Řízení průběhu hry}
Po příkazu ke spuštění hry se nejdříve ověří, zda jsou k dispozici alespoň dva hráči. Pokud tomu tak není, hra se nezahájí a uživatelé jsou tímto informování indikační svítivou diodou (viz. tabulka \ref{tab:serverLED}).

V případě, že je k dispozici dostatečný počet hráčů, je náhodně vybrán (funkce \texttt{random()}) první hráč a všem hráčům se rozešle pole \textit{board} s vyplněným herní kódem (1) a číslem hráče (metoda odeslání popsána v kapitole \ref{sec:comm_server-client}). Jednotliví hráči porovnají pořadí svých IP adres v poli \textit{board} a zjištěný index porovnají s číslem hráče vyplněným na pozici 91 v poli \textit{board}, hráč, který nalezne shodu, je na tahu, ostatní vyčkávají.

Server nyní čeká dokud mu hrající hráč neodešle pozici, kterou si přeje vyplnit. Komunikace klienta se serverem se řídí podle pravidel popsaných v kapitole \ref{sec:comm_client-server}. Přeskočit daného hráče lze pomocí stisku zeleného tlačítka na serveru. V případě, že si hrající uživatel přeje vyplnit danou pozici, stiskne patřičné místo na displeji. Klient zkontroluje, zda je daná pozice volná a pokud ano okamžitě vyplní žeton a následně odešle informaci o vyplnění serveru. Okamžité vykreslení žetonu před potvrzení serverem je implementováno, protože při delší odezvě serveru může docházet ke zpoždění vykreslení (v poměru s dotykem), což je uživatelsky nekomfortní. Pokud by klient odeslal serveru žádost na vyplnění již obsazeného pole, server odešle všem klientům pole \textit{board} a na pozici hrajícího hráče (\texttt{board[91]}) ponechá sejného hráče. Na straně klienta se při vykreslování neobsazené pozice vykreslí žeton s barvou pozadí (černou), čímž se vymaže nesprávně umístěný žeton.

Pokud výběr a obsazení pozice proběhlo v pořádku, server odešle všem pole s kódem (2) - pouze překresli obrazovku, nikdo nehraje. Následně server zkontroluje stav hry: zda je počet odehraných kolm menší než počet pozic v poli (jinak odešle kód (100) - remíza), zkontroluje zda hráč, který odehrál kolo nespojil dostatečný počet žetonů (definováno v proměnné \texttt{crossNum}, defaultně pět). V případě, že by spojil a mět tak nárok na výhru, všem se odešle pole \textit{borad} s kóde 10x (\textit{x} je číslo hráče, který vyhrál) a klient vypíše na displej danou informaci. V opačném případě hra pokračuje, server vyplní na pozici hrajícího hráče (\texttt{board[91]}) číslo dalšího hráče, zvětší počet kol o jedna (\texttt{board[93]}) a všem hráčům opět odešle pole s kódem (1).

Celý tento systém se opakuje, dokud nějaký hráč nevyhrál, hra neskončila remízou nebo hra není ukončena ze strany serveru.

Neustále se také ověřuje, zda se nějaký hráč během hry neodpojil, pokud ano, jeho IP adresa je smazána z pole a na straně serveru nezabírá místo v poli aktivních klientů (\texttt{clients[]}). Během hry se však žádný klient nemůže připojit ani v případě, že ve hře při jejím započetí byl. Žetony odpojených klientů zůstávají ve hře. Pokud by došlo k odpojení všech klientů, server hru ukončí.
