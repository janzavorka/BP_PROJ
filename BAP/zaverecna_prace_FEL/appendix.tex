\appendix
\section{Soubor s daty}
Součástí této práce je i soubor \uv{zavorja4\_BP\_priloha.zip} s zdrojovými kódy a dalšími daty. Soubor má následující strukturu:
\vspace*{.5cm}
\noindent
%
\setlength{\DTbaselineskip}{20pt}
\DTsetlength{1em}{2em}{0.1em}{1pt}{4pt}
\dirtree{%
.1 zavorja4\_BP\_priloha  .
.2 {bakalarska\_prace\_JanZavorka.pdf}.
.2 soubory\_krabicek.
.3 klient.
.4 {case\_bottom.stl}.
.4 {case\_top.stl}.
.3 server.
.4 {case\_bottom.stl}.
.4 {case\_top.stl}.
.2 zdrojove\_kody.
.3 piskvorky\_MP\_client \makebox[1cm]{\dotfill} \begin{minipage}[t]{5cm}\fontfamily{cmr}\selectfont\textit{Zdrojové kódy pro klienta, pro správné fungování musí být zachován název složky a souborů uvnitř}{.}
\end{minipage}
.
.4 {piskvorky\_MP\_client.ino}.
.4 {displayControl.ino}.
.4 {gameControl.ino}.
.4 {communication.ino}.
.3 piskvorky\_MP\_server\makebox[1cm]{\dotfill} \begin{minipage}[t]{5cm}\fontfamily{cmr}\selectfont\textit{Zdrojové kódy pro server, pro správné fungování musí být zachován název složky a souborů uvnitř}{.}
\end{minipage}.
.4 {piskvorky\_MP\_server.ino}.
.4 {boardControl.ino}.
.4 {gameControl.ino}.
.4 {communication.ino}.
.4 {indicationLED.ino}.
.4 {SerialControl.ino}.
}%
%
